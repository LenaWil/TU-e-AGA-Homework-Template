\documentclass[
  % all of the below options are optional and can be left out
  % course name (default: 2IL50 Data Structures)
  course = {{2IL50 Data Structures}},
  % quartile (default: 3)
  quartile = {{3}},
  % assignment number/name (default: 1)
  assignment = 1,
  % student name (default: Some One)
  name = {{Some One ; Other Person}},
  % student number, NOT S-number (default: 0123456)
  studentnumber = {{0123456 ; 0314159}},
  % student email (default: s.one@student.tue.nl)
  email = {{s.one@student.tue.nl ; o.person@student.tue.nl}},
  % first exercise number (default: 1)
  firstexercise = 1
]{aga-homework}


\begin{document}

\noindent {\footnotesize Note: This is by no means a full and complete introduction to \LaTeX{}. Its purpose is to introduce some of the symbols and notation used in various courses, and demonstrate how you can accomplish common tasks in \LaTeX{}. We encourage you to use this document as a template for your solutions.}


\exercise

\subexercise
My solution to this exercise is a bunch of \textbf{logical symbols}:
\begin{center}
  \begin{tabular}{>{\ttfamily}l c @{\hspace{3em}} >{\ttfamily}l c}
    \toprule
    \textbackslash{}lor & $\lor$ &
    \textbackslash{}land & $\land$ \\
    \textbackslash{}Rightarrow & $\Rightarrow$ &
    \textbackslash{}Leftrightarrow & $\Leftrightarrow$ \\
    \textbackslash{}neg & $\neg$ &
    \textbackslash{}neq & $\neq$ \\
    \textbackslash{}equiv & $\equiv$ &
    \textbackslash{}not\textbackslash{}equiv & $\not\equiv$ \\
    \bottomrule
  \end{tabular}
\end{center}
Although we usually use them inline: $\frac{x}{2} - y > 4 \Rightarrow x > 2(y + 4)$, or in a block:
\[
  \frac{x}{2} - y > 4 \Rightarrow x > 2(y + 4).
\]

\subexercise
Can't forget the \emph{set operations} and so forth: $P \cup \emptyset \not\subseteq Q$.
\begin{center}
  \begin{tabular}{>{\ttfamily}l c @{\hspace{3em}} >{\ttfamily}l c}
    \toprule
    \textbackslash{}cap & $\cap$ &
    \textbackslash{}cup & $\cup$ \\
    \textbackslash{}setminus & $\setminus$ &
    \texttt{A\textasciicircum{}\textbackslash{}text\{c\}} & $A^\text{c}$ \\
    \textbackslash{}mathcal\{P\} & $\mathcal{P}$ &
    \textbackslash{}mathcal\{U\} & $\mathcal{U}$ \\
    \textbackslash{}in & $\in$ &
    \textbackslash{}not\textbackslash{}in & $\not\in$ \\
    \textbackslash{}subseteq & $\subseteq$ &
    \textbackslash{}not\textbackslash{}subseteq & $\not\subseteq$ \\
    \textbackslash{}subset & $\subset$ &
    \textbackslash{}not\textbackslash{}subset & $\not\subset$ \\
    \textbackslash{}emptyset & $\emptyset$ &
    \textbackslash{}varnothing & $\varnothing$ \\
    \bottomrule
  \end{tabular}
\end{center}
You may also want to define sets explicitly: $4 \not\in \{1, 2, 7\}$.

\subexercise
Some basic mathematical notation: normally, one needs to write \texttt{\textbackslash{}mathbb\{R\}} to get~$\R$, but we have defined shortcuts for you: \texttt{\textbackslash{}R} works!
\[ \R \quad \N \quad \Z \quad \Q \quad < \quad > \quad \leq \quad \geq \quad \neq \quad x^2 \quad \sqrt{y} \]
\[ 3 \cdot 3 = 3^2 = 9 \]

\subexercise
Here's a math table-like environment for setting up a series of derivations.
\begin{eqnarray*}
  && T \\
  &\equiv& \{ \text{rule to be proven } \Rightarrow \} \\
  && ? \\
  &\equiv& \{ \text{rule to be proven } \wedge \} \\
  && F
\end{eqnarray*}

But if you don't want to use (a lot of) math inside a table, a normal table also works:
\begin{center}
  \begin{tabular}{lcr}
    \toprule
    left & center & right \\
    \midrule
    l & c & r \\
    $x+y$ & $\equiv$ & $y + x$ \\
    \bottomrule
  \end{tabular}
  %
  \hspace{5em}
  %
  \begin{tabular}{rl}
    & $a \vee b$ \\
    & $\neg a$ \\
    \hline
    $\therefore$ & $b$
  \end{tabular}
\end{center}


\exercise

\subexercise Regular expressions can be written as: $(a + \varepsilon)^* (a+b) ab$.
\subexercise Useful for writing about Turing machines, but also for concluding a proof: $\square$.

In fact, we can write a proof as follows.

\begin{proof}
  It can easily be seen that this statement is concluded with a \texttt{\textbackslash{}qed} symbol.
\end{proof}


\exercise

We can specify an algorithm in pseudo-code as follows.

\begin{algorithm}[H]
  \caption{MyIncredibleAlgorithm($A, v$)}
  \KwIn{an array $A$ of $n$ numbers and a number $v$}
  \KwOut{an index $i$ such that $A[i] = 42$, or \textsc{NotFound} if no such index exists}

  $i \leftarrow 1$ \;
  \While{$i \leq n$ \And $A[i] \neq 42$}{
    $A[i].\textit{update()}$ \;
    $i \leftarrow i + v^2$
  }
  \tcc{the following loop runs over all of $A$}
  \For(\tcp*[f]{use a different indicator variable here}){$j \leftarrow 1$ \To $n$}{
    something useful \tcp*{you can write text if that's more succinct}
    \If{$j$ is even}{
      $A[j].\textit{flip()}$
    }
  }
  \leIf{$i > n$}{
    \Return{\textsc{NotFound}}
  }{
    \Return{$i$}
  }
\end{algorithm}

\noindent For analysis, you may want to know $O, \Theta, \Omega(n \log n)$ (not to be confused with $o, \omega$).


\exercise

A figure (such as Figure~\ref{fig:squares}) can be inserted, PDF format is preferred.

\begin{figure}[b] % The letters between square brackets tell LaTeX where the figure can be placed:
  % `h` means 'right here';
  % `b` means 'at the bottom of the page';
  % `t` means 'at the top of the page';
  % `p` means 'at a separate page (optionally with other figures if that fits).
  % Specifying only `h` will not work, LaTeX will automatically change that to `ht`.
  \begin{center}
    \includegraphics{squares}
    \caption{Three axis-aligned squares.}
    \label{fig:squares}
  \end{center}
\end{figure}

To insert a figure precisely where you want it to appear, you can leave away the \texttt{figure} environment. You will then however not be able to add a caption or refer to it.

\begin{center}
  \includegraphics{squares}
\end{center}

The caption of Figure~\ref{fig:squares} can be left away completely, but you will not be able to refer to the figure anymore if you do that; the \texttt{\textbackslash{}label} only works right after a \texttt{\textbackslash{}caption}. Note how \LaTeX{} can automatically fill in the figure number for you, making it easy to refer to figures. The same approach works for tables, although it is common practice to put the caption before the table instead of below it; this can be seen in Table~\ref{tab:courses}.

\begin{table}[bt]
  \caption{Courses taught by members of the Applied Geometric Algorithms group.}
  \label{tab:courses}
  \begin{center}
    \begin{tabular}{ll}
      \toprule
      \textbf{Course code} & \textbf{Course name} \\
      \midrule
      2IL50 & Data Structures \\
      JBI020 & Foundations of Computing \\
      \bottomrule
    \end{tabular}
  \end{center}
\end{table}

\subexercise
As you can see, you can put text before the first subexercise as well, if you wish.

\exercise[6]
\subexercise[b]
You can explicitly state the exercise too. This is useful if you cannot answer all questions.

\subexercise
Further subexercises can be numbered automatically again.

\exercise
The numbering will continue where it left off.

\end{document}
